%!TEX root = ../dissertation.tex


\section{Deterministic Data}
\label{ddata}

Forecasts and corresponding observations must be readily available for one to develop a statistical post-processing method for calibration of numerical model forecasts.
Unfortunately, observational datasets of many RCEs (e.g., tornadoes, wind damage, hail swaths, etc) are riddled with inaccuracies and deficiencies \citep{Doswell1988, Weiss2002, Trapp2006, Ortega2009} that prevent their use in rigorous statistical post-processing.
Precipitation, however, is one of the best documented observational fields in meteorology, and it is available as direct output from numerical models.
Thus, to develop a prototype method of calibration for RCEs, accumulated precipitation was selected as the predictand.
Here, accumulations greater than or equal to \mbox{25.4 mm} in \mbox{6 hr} are considered rare events as this threshold is met on fewer than 0.36\% of all grid points\footnote{If one considers a forecast at a single grid point as a single forecasting occasion, then \mbox{25.4 mm} in \mbox{6 hr} meets the definition of a rare event posed by \cite{Murphy1991}.} (\mbox{Figure \ref{single_25quant}}).
Subsequently, the proposed method was applied to forecasts of accumulations greater than or equal to \mbox{12.7 mm} in \mbox{6 hr} to test the method against a less rare event\footnote{Forecasts of at least \mbox{12.7 mm} in \mbox{6 hr} occurred on approximately 1.3\% of all grid points.} (\mbox{Figure \ref{single_12quant}}).


Model forecasts and observations of precipitation were obtained for the 48-month period 01 April 2007 through 31 March 2011 and subdivided into two categories: training data and verification data.
Forecasts and observations during the time period 01 April 2007 through 31 March 2010 (36 months) were used in the training dataset, with the remaining 12 months used for testing and verification of the proposed method.




\subsection{Deterministic Model Forecasts}
\label{dmodel}

Model forecasts were taken from the \mbox{4 km} grid-length Weather Research and Forecasting -- Advanced Research WRF (WRF-ARW) model configuration \citep{WRFV3} run daily at the National Oceanic and Atmospheric Administration (NOAA) National Severe Storms Laboratory (NSSL).
The NSSL produces numerical weather prediction forecasts from the WRF-ARW model as part of an ongoing collaborative effort with the NOAA Storm Prediction Center (SPC).
Model forecasts are produced daily, integrated out to 36 hours, using 00 UTC initial and lateral boundary conditions from the operational North American Mesoscale model \citep{NAMnWRF-NMM}, over a CONUS domain.
Information on the configuration is provided in \cite{Kain2010}.\footnote{Images of output from the WRF forecasts generated at the NSSL, hereafter NSSL-WRF, can be found at \url{http://www.nssl.noaa.gov/wrf}}




\subsection{Observations}
\label{observations}

Observations were taken from the NOAA National Centers for Environmental Prediction (NCEP) Stage IV national quantitative precipitation estimate analysis.
The stage IV analyses are based on the multi-sensor hourly/6-hourly `Stage III' analyses (on local \mbox{4.7 km} polar-stereographic grids) produced by the 12 River Forecast Centers in the CONUS.
NCEP mosaics the Stage III into a national product (the Stage IV analyses) available in hourly, 6-hourly, and 24-hourly (accumulated from the 6-hourly) intervals.
\cite{StageIV} describe further details of these analyses\footnote{Archives of the Stage IV dataset can be found at \url{http://data/eol.ucar.edu/codiac/dss/id=21.093}.}.




\subsection{Processing}
\label{dprocessing}

Diagnostic analyses were conducted on the Stage IV grid, requiring interpolation of the NSSL WRF-ARW\footnote{Hereafter referred to as NSSL-WRF.} output.
The program \emph{copygb}\footnote{Available at \url{http://www.cpc.ncep.noaa.gov/products/wesley/copygb.html.}} was used for the interpolation and domain-wide total liquid volume was conserved.
Six-hour accumulation periods were used for both model forecasts and the Stage IV analyses, with model forecasts coming from the \mbox{12-36 hr} integration periods ending at 18, 00, 06, and 12 UTC.
A mask was applied to both the NSSL-WRF forecasts and Stage IV analyses to limit the region studied to CONUS and near-CONUS areas east of the Rocky Mountains \mbox{(Fig. \ref{domain})}.


Over the course of the three year training dataset and the one year verification dataset there were occasional \mbox{6 hr} time periods where either the NSSL-WRF data or the Stage IV analyses were unavailable.
If either the Stage IV analysis or the NSSL-WRF data were unavailable for a given \mbox{6 hr} time period, that particular time period was removed from consideration.
Additionally, some of the Stage IV analyses contain missing data over portions of the domain shown in \mbox{Figure \ref{domain}}.
In these situations, the missing grid points from the Stage IV analyses were masked on the corresponding NSSL-WRF grid, and the training and verification of the proposed method was carried out on the remaining grid points.
