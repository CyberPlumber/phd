%!TEX root = ../dissertation.tex


\section{Ensemble Data}
\label{edata}

As discussed at the beginning of \mbox{chapter \ref{deterministic}}, in order to develop a statistical post-processing method for calibration of numerical weather prediction forecasts, both forecasts and observations must be readily available.
Unfortunately, as mentioned in the Introduction, ``\dots there is a limited database of forecasts for RCEs, making robust statistical techniques difficult\dots''
As limited as databases of deterministic CAM forecasts are, the availability of ensembles of CAM forecasts is even worse.


Similar to CAM forecasts, most ensemble CAM (eCAM) forecasts have been short-term initiatives, produced in support of various field programs and experiments.
The makeup of these eCAMs are often modified year-to-year to adapt to the changing needs of the various field programs and experiments.
This makes finding a long running, consistent ensemble extremely difficult.


One of the largest collections of eCAM forecasts has been produced by the University of Oklahoma's Center for the Analysis and Prediction of Storms (CAPS).
Since 2007, CAPS has been producing CAM forecasts in support of the Hazardous Weather Testbed's annual Spring Forecast Experiment.
Typically the ensemble configuration changes year-to-year based on the results of previous years.
However, in 2011 a conscious effort was made to ensure a subset of that year's CAM forecasts would be produced using the same configuration as the previous year.
This resulted in the creation of a two year, 15-member eCAM dataset that is used here.




\subsection{Ensemble Forecasts}
\label{emodel}





\subsection{Observations}
\label{eobservations}

As was the case for the deterministic model calibration, NCEP's Stage IV national quantitative precipitation estimate analysis was chosen as the verification dataset\footnote{See \mbox{section \ref{observations}} for a description of this dataset.}.
Similarly, as was necessary with the deterministic forecasts, the ensemble forecasts have been interpolated to the Stage IV grid, and all diagnostics and analysis were conducted on this grid.
Furthermore, the same mask shown in \mbox{Figure \ref{domain}} was used to limit the analysis to areas east of the Rocky Mountains and near land.


